\documentclass[dvipdfmx,11pt]{jsarticle}
\usepackage{mathtools}
\usepackage{amssymb}
\usepackage{listings,jvlisting}
\usepackage{bm}

\usepackage{calc}
\usepackage{enumitem}
\lstset{
  basicstyle={\ttfamily},
  identifierstyle={\small},
  commentstyle={\smallitshape},
  keywordstyle={\small\bfseries},
  ndkeywordstyle={\small},
  stringstyle={\small\ttfamily},
  frame={tb},
  breaklines=true,
  columns=[l]{fullflexible},
  numbers=left,
  xrightmargin=0zw,
  xleftmargin=3zw,
  numberstyle={\scriptsize},
  stepnumber=1,
  numbersep=1zw,
  lineskip=-0.5ex
}
\renewcommand{\lstlistingname}{アルゴリズム}

% \newcommand{\argmin}{\mathop{\rm argmin}\limits}

% \renewcommand{\thepart}{\arabic{part}}
% \renewcommand{\thesection}{\thepart.\arabic{section}}
% \renewcommand{\thesubsection}{\thesection.\arabic{subsection}}
\newcommand{\noise}{\mathrm{noise}}
\newcommand{\depth}{D}
\newcommand{\dist}{d}
\newcommand{\NNR}{\mathbb{R}_+}
\newcommand{\mywidth}{\widthof{\bfseries データセットのサイズ }}
\newcommand{\myHspace}{\mathcal{T}_\mathcal{F}}
\newcommand{\Dmax}{\depth_\mathrm{max}}
\newcommand{\fdim}{M}
\numberwithin{equation}{part}

\begin{document}
\section{主問題}

\subsection{所与の条件}
\begin{description}[leftmargin=!,labelwidth=\mywidth]
  \item[特徴空間の次元] $\fdim\in\mathbb{N}$
  \item[特徴空間(距離空間)]
  $
    (\mathcal{F},\dist)
    \quad\mathcal{F}=\mathcal{F}_1\times\mathcal{F}_2\times
    \dots\times\mathcal{F}_\fdim,
    \quad\dist:\mathcal{F}\times\mathcal{F}\to\NNR
  $

  % \item[ラベル数] $K\in\mathbb{N}$
  \item[ラベル空間] $\mathcal{L}$%=\{0,1\}$
  \item[学習済み分類器] $f:\mathcal{F}\to \mathcal{L}$
\end{description}
\begin{description}[leftmargin=!,labelwidth=\mywidth]
  \item[目的データ] $x^*\in\mathcal{F}$
  \item[深さ制約] $\Dmax\in\mathbb{N}$
  \item[精度制約] $A_\mathrm{min}\in[0,1]$
\end{description}
\subsection{決定木}
\begin{description}[leftmargin=!,labelwidth=\mywidth]
\item[決定木] $t:\mathcal{F}\to \mathcal{L}$
\item[木の深さ] $\depth(t):\mathcal{T}_\mathcal{F}\to \mathbb{N}$
\item[仮説空間] $\myHspace$ --- 特徴空間$\mathcal{F}$において, 可能な決定木の集合
\item $\myHspace(\Dmax)=\{t\in\myHspace\mid \depth(t)\le\Dmax\}$
\end{description}

\subsection{変数}
\begin{description}[leftmargin=!,labelwidth=\mywidth]
  \item[近傍半径] $r\in\NNR$
\end{description}

\subsection{関数など}
\begin{description}[leftmargin=!,labelwidth=\mywidth]
  \item[近傍]
  $V_{x^*}(r)=\left\{x\in\mathcal{F}\mid\dist(x,x^*)\le r\right\}$
\end{description}
\begin{description}[leftmargin=!,labelwidth=\mywidth]
  \item[ノイズ集合]
  $
    \noise(r):\NNR\to2^{F}\quad
    \forall r\in\NNR\ ;\ \noise(r)\subseteq
    V_{x^*}(r)\wedge\noise(r)\text{ is finite.}
  $
\end{description}
\begin{description}[leftmargin=!,labelwidth=\mywidth]
  \item[近似精度]
  \begin{flalign*}
    &A(t,r)=\frac{1}{|\noise(r)|}\sum_{x\in\noise(r)}\mathbb{I}(t(x)=f(x))&
  \end{flalign*}
\end{description}

\subsection{問題}
$\exists t\in \myHspace(\Dmax)\ ;\ A(t,r)\ge A_\mathrm{min}$
を満足する最大の近傍半径$r\in\NNR$を求める.

\newpage
\section{固定されたデータセットの場合}

\subsection{所与の条件}
\begin{description}[leftmargin=!,labelwidth=\mywidth]
  \item[特徴空間の次元] $\fdim\in\mathbb{N}$
  \item[特徴空間]
  $
    \mathcal{F}=\mathcal{F}_1\times\mathcal{F}_2\times
    \dots\times\mathcal{F}_\fdim
  $
  % \item[ラベル数] $K\in\mathbb{N}$
  \item[ラベル空間] $\mathcal{L}$%$=\{0,1\}$
  \item[\underline{データセットのサイズ}] $N\in\mathbb{N}$
  \item[\underline{データセット}]
  $X=\left\{x_i\in\mathcal{F}\right\}_{i=1}^N,
  \ \ Y=\left\{y_i\in\mathcal{L}\right\}_{i=1}^N$
\end{description}
\begin{description}[leftmargin=!,labelwidth=\mywidth]
  \item[目的データ] $x^*\in X$
  \item[深さ制約] $\depth_\mathrm{max}\in\mathbb{N}$
  \item[精度制約] $A_\mathrm{min}\in[0,1]$
\end{description}
\subsection{決定木}
\begin{description}[leftmargin=!,labelwidth=\mywidth]
\item[決定木]   $t:\mathcal{F}\to \mathcal{L}$
\item[木の深さ] $\depth(t):\mathcal{T}_\mathcal{F}\to \mathbb{N}$
\item[仮説空間] $\mathcal{T}_\mathcal{F}$
--- 特徴空間$\mathcal{F}$において, 可能な決定木の集合
\item $\myHspace(\Dmax)=\{t\in\myHspace\mid \depth(t)\le\Dmax\}$
\end{description}

\subsection{関数など}
\begin{description}[leftmargin=!,labelwidth=\mywidth]
  \item[\underline{近似精度}]
  \begin{flalign*}
    &A_{X,Y}(t)=\frac{1}{N}\sum_{i=1}^{N}\mathbb{I}(y_i=t(x_i))&
  \end{flalign*}
\end{description}

\subsection{問題}
$A_{X,Y}(t)\ge A_\mathrm{min}$
を満足する$t\in \myHspace(\Dmax)$が存在するか否かを判定する.

% \newpage
% \section{固定されたデータセットの場合}

% \subsection{所与の条件}
% \begin{description}[leftmargin=!,labelwidth=\mywidth]
%   %\item[特徴量の数] $M\in\mathbb{N}$
%   \item[特徴空間(距離空間)] $(\mathcal{F},\dist)
%   \quad\dist:\mathcal{F}\times\mathcal{F}\to\NNR$%=\{0,1\}^M$
%   % \item[ラベル数] $K\in\mathbb{N}$
%   \item[ラベル空間] $\mathcal{L}$%$=\{0,1\}$
%   \item[\underline{データセットのサイズ}] $N\in\mathbb{N}$
%   \item[\underline{データセット}] $X=\left\{x_i\in\mathcal{F}\right\}_{i=1}^N$
%   \item[学習済み分類器] $f:\mathcal{F}\to \mathcal{L}$
% \end{description}
% \begin{description}[leftmargin=!,labelwidth=\mywidth]
%   \item[目的データ] $x^*\in X$
%   \item[深さ制約] $\depth_\mathrm{max}\in\mathbb{N}$
%   \item[精度制約] $A_\mathrm{min}\in[0,1]$
% \end{description}
% \subsection{決定木}
% \begin{description}[leftmargin=!,labelwidth=\mywidth]
%   \item[仮説空間] $\mathcal{T}_\mathcal{F}$
%   --- 特徴空間$\mathcal{F}$において, 可能な決定木の集合
% \item[決定木] $t\in\mathcal{T}_\mathcal{F}$
% \item $t:\mathcal{F}\to \mathcal{L}$
% \item[木の深さ] $\depth(t):\mathcal{T}_\mathcal{F}\to \mathbb{N}$
% \end{description}

% \subsection{変数}
% \begin{description}[leftmargin=!,labelwidth=\mywidth]
%   \item[近傍半径] $r\in\NNR$
% \end{description}

% \subsection{関数など}
% \begin{description}[leftmargin=!,labelwidth=\mywidth]
%   \item[\underline{ノイズ集合}]
%   $
%     \noise_{x^*}(r):\NNR\to2^\mathcal{F}\quad\quad
%     \noise_{x^*}(r)=\left\{x\in X\mid\dist(x-x^*)\le r\right\}\subseteq X
%   $
% \end{description}
% \begin{description}[leftmargin=!,labelwidth=\mywidth]
%   \item[\underline{近似精度}]
%   $A_{f,x^*}(t,r):\mathcal{T}_\mathcal{F}\times\NNR\to[0,1]$
%   $$
%     A_{f,x^*}(t,r)=\frac{1}{|\noise_{x^*}(r)|}
%     \sum_{x\in\noise_{x^*}(r)}\mathbb{I}(t(x)=f(x))
%   $$
% \end{description}

% \subsection{問題}
% $
%   \exists t\in \mathcal{T}_\mathcal{F}\ ;
%   \ \depth(t)\le \depth_\mathrm{max}\wedge
%   A_{f,x^*}(t,r)\ge A_\mathrm{min}
% $
% を満足する最大の近傍半径$r\in\NNR$を求める.

\end{document}